\section{Flow Chart}
\begin{frame}{Flow Chart}
    %For Flow Chart declaration
    \tikzstyle{startstop} = [ellipse, minimum width = 3cm, minimum height = 1cm, text centered, draw = black, fill = red!30]
    \tikzstyle{io} = [trapezium, trapezium left angle = 70, trapezium right angle = 110, minimum width = 3cm, minimum height = 1cm, text centered, draw = black, fill = blue!30]
    \tikzstyle{process} = [rectangle, minimum width = 3cm, minimum height = 1cm, text centered, draw = black, fill = orange!30]
    \tikzstyle{decision} = [diamond, aspect = 4, minimum width = 3cm, minimum height = 1cm, text centered, draw = black, fill = green!30 ]
    \tikzstyle{arrow} = [thick, ->, >=stealth]
    
    \begin{figure}
        \centering
        \begin{tikzpicture}
        \node[startstop] (start) {Start};
        \node[process, below of = start, yshift = -0.5cm] (p1) {Process - 1};
        \node[io, left of= p1, xshift = -3.5cm](io1) {Documents};
        \node[decision, below of = p1, yshift = -0.5cm] (d1) {Decision};
        \node[process, below of= d1, yshift = -0.5cm] (pA) {Process A};
        \node[process, below of= d1, yshift = -0.5cm, xshift = 3.5cm] (pB) {Process B};
        \node[startstop, below of = pA, yshift = -0.5cm] (end) {END};
        \draw[arrow] (start) -- (p1);
        \draw[arrow] (io1) -- (p1);
        \draw[arrow] (p1) -- (d1);
        \draw[arrow] (d1) -- node[right] {YES} (pA);
        \draw[arrow] (d1) -| node[above] {NO} (pB);
        \draw[arrow] (pA) -- (end);
        \draw[arrow] (pB) |- (end);
        
        \end{tikzpicture}
        \caption{Flow Chart}
        \label{fig:flow_chart}
    \end{figure}
\end{frame}