\documentclass{beamer}
\usepackage{graphicx}
\usetheme{Madrid}

\title{Induction}
\subtitle{Strong Induction \& Weak Induction}
\author{Mobaswirul Islam}
\institute[CSE, BUET]{Computer Science \& Engineering, Bangladesh University of Engineering \& Technology}
\date{\today}
\logo{\includegraphics[height=.4cm]{logo_2.jpg}}

\begin{document}

\begin{frame}{}
    \titlepage
\end{frame}

\section{Introduction}
\begin{frame}{What is Induction?}
    \begin{itemize}
        \item Proof technique \pause
        \item Simple yet efficient method \pause
        \item Proves a statement for a set of numbers
    \end{itemize}
\end{frame}

\section{Steps \& Types}

\begin{frame}{Steps \& Types}
\begin{columns}
    \begin{column}{0.3\textwidth}
    \textbf{Steps}
    \begin{itemize}
        \item Base Step
        \item Inductive Step
    \end{itemize}
    \end{column}
    \begin{column}{0.4\textwidth}
    \pause
    \textbf{Types}
    \begin{itemize}
        \item Weak Induction (Regular Induction)
        \item Inductive Step
    \end{itemize}
    \end{column}
\end{columns}
\end{frame}

\section{Principles}
\begin{frame}{Principles of \textit{Weak Induction}}
    \begin{enumerate}
        \item \textbf{Base Step}: verify $P(1) \implies P(0)$ \pause
        \item \textbf{Inductive Step}: prove $P(n) \implies P(n+1)$
    \end{enumerate}
\end{frame}

\begin{frame}{Principles of \textit{Strong Induction}}
    \begin{enumerate}
        \item \textbf{Base Step}: verify $P(1) \implies P(0)$
        \item \textbf{Inductive Step}: prove 
        $\forall k \in N: k \leq n, P(k) \implies P(n+1)$
    \end{enumerate}
\end{frame}

\section{Equivalence}
\begin{frame}{Equivalence of Weak \& Strong Induction}
    \setbeamercovered{transparent}
    \begin{itemize}
        \item<1,3> If a statement can be proven by \textbf{Weak Induction}, it can be proven by \textbf{Strong Induction}
        \item<2,3> If a statement can be proven by \textbf{Strong Induction}, it can be proven by \textbf{Weak Induction}
        \item<3> First one is left for exercise.
    \end{itemize}
    \setbeamercovered{invisible}
\end{frame}

\section{Strong to Weak}
\begin{frame}{If provable by Strong, can be proven by Weak}
Let, a statement be P. $\forall n \in N, P(n)$
If P can be proven by \textbf{Strong Induction}
\begin{enumerate}
    \item $P(1)$ is true
    \item $\forall k \in N : k \leq P(k) \implies P(n+1)$
    We are to prove, $P$ can be proven using \textit{Weak Induction}
\end{enumerate}
\end{frame}

\begin{frame}{If provable by Strong, can be proven by Weak}
    Define, 
    \begin{equation*} \label{eq1}
        \begin{split}
            Q(n) & = \forall k, n \in N : k \leq n, P(k) \\
            & = P(1) \land P(2) \land P(3) \ldots P(n)
\end{split}
\end{equation*}

It is Obvious, $Q(1)$ is true, as $Q(1) = P(1)$\\
Rewriting we get, $\forall n \in N, Q(n) \implies P(n+1)$\\
As, $(A\implies B) \land (A\implies C) \implies (A \implies (B \land C))$\\
So, we get,
\begin{equation*}
    \begin{split}
        Q(n) &\implies Q(n) \land P(n+1)\\
        &\implies P(1) \land P(2) \ldots P(n) \land P(n+1) \\
        Q(n) &\implies Q(n+1)\\
    \end{split}
\end{equation*}
\end{frame}

\begin{frame}{If provable by Strong, can be proven by Weak}

Now, 
$$Q(1) \land (\forall n \in N, Q(n) \implies Q(n+1)) \implies \forall n \in N, Q(n)$$
which is the weak inductive principle
\\
From our assumption, $Q(n)$ can never be true without $P(n)$\\
So, $Q(n) \implies P(n)$\\
Therefore, the statement $P(n)$ for all n is just proved by using weak induction.
\end{frame}


\end{document}